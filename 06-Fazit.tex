\section{Fazit}
Anhand des Modells konnte dargestellt werden, dass die unterschiedlichen Szenarien zum Teil und insbesondere in Abhängigkeit der Treibhausreduzierungsziele, stark voneinander abweichen. Der Anwender kann sich mit Hilfe dieses Programms ein Bild über die erforderlichen Maßnahmen bis zur 100 \%igen CO2-Neutralität machen. 
Festzuhalten ist der überproportionale Bedarf an Energie, der die Grundlage für das zu entwickelnde Energiesystem darstellt. Daneben stellen die PtX-Anlagen ein weiteres wichtiges Element zur Sektorenkopplung dar, um den Wasserstoffbedarf neben Importen selber zu decken. 
Die ambitionierten CO2 Ziele der Bundesregierung lassen aus heutiger Sicht eine nicht Erreichung mutmaßen. Umso wichtiger erscheinen die daraus gewonnenen Erkenntnisse, um den Wert annähernd zu erreichen. Der Stromerzeugungsmix wird als zwingende Voraussetzung ermittelt, so dass der Ausbau der erneuerbaren Energien einen sehr hohen Stellenwert hat. Somit wird die Zubaurate an EE in den Folgejahren signifikant steigen müssen. 
Neben der Erzeugung kann der Speicherung von H2 und Energie eine tragende Rolle zugesprochen werden. Die Erschließung der Speicher an das Energiesystem in Deutschland ermöglichen eine Langzeitspeicherung und unterstützen die Treibhausreduktionsziele. 

\newpage
Anhand der durchgeführten Simulation eines nationalen Energiesystems können erste Erkenntnisse zu einem Szenario mit einem Energiebedarf von Strom und Wasserstoff und Verwendung aller möglicher Komponenten zur Erzeugung erneuerbarer Energie und Übertragungsmöglichkeiten gezogen werden.

\todo{Etwas von oben übernehmen}

\todo{Welche Reduktion kann unser Energiesystem möglichst gut erreichen? Abwägung Reduktion/Kosten}

\todo{Was würde eine 100 \% Reduktion bedeuten? Ist das machbar?}


Die gewonnenen Erkenntnisse können mit anderen Simulationen, die andere Komponenten verwenden, hinsichtlich ihrer Gesamtkosten und Machbarkeit verglichen werden.