\section{Fazit}
Anhand des Modells konnte dargestellt werden, dass die unterschiedlichen Szenarien zum Teil und insbesondere in Abhängigkeit der Treibhausreduzierungsziele, stark voneinander abweichen. Der Anwender kann sich mit Hilfe dieses Programms ein Bild über die erforderlichen Maßnahmen bis zur 100 \%igen CO2-Neutralität machen. 
Festzuhalten ist der überproportionale Bedarf an Energie, der die Grundlage für das zu entwickelnde Energiesystem darstellt. Daneben stellen die PtX-Anlagen ein weiteres wichtiges Element zur Sektorenkopplung dar, um den Wasserstoffbedarf neben Importen selber zu decken. 
Die ambitionierten CO2 Ziele der Bundesregierung lassen aus heutiger Sicht eine nicht Erreichung mutmaßen. Umso wichtiger erscheinen die daraus gewonnenen Erkenntnisse, um den Wert annähernd zu erreichen. Der Stromerzeugungsmix wird als zwingende Voraussetzung ermittelt, so dass der Ausbau der erneuerbaren Energien einen sehr hohen Stellenwert hat. Somit wird die Zubaurate an EE in den Folgejahren signifikant steigen müssen. 
Neben der Erzeugung kann der Speicherung von H2 und Energie eine tragende Rolle zugesprochen werden. Die Erschließung der Speicher an das Energiesystem in Deutschland ermöglichen eine Langzeitspeicherung und unterstützen die Treibhausreduktionsziele. 

\newpage
Anhand der durchgeführten Simulation eines nationalen Energiesystems können erste Erkenntnisse zu einem Szenario mit einem Energiebedarf von Strom und Wasserstoff und Verwendung aller möglicher Komponenten zur Erzeugung erneuerbarer Energie und Übertragungsmöglichkeiten gezogen werden.

Stichpunkte:
\begin{itemize}
  \item Der erforderliche Rückbau von Methan zur Senkung von CO2 kann durch andere Erzeugungsanlagen EE und Biogas ausgeglichen werden
  \item Biogas ist hierbei besonders wichtig, da dies genau wie Speicher dem Energiesystem Flexibilität verleiht
  \item In allen Reduktionsfällen kann auf Power-to-Gas-to-Power verzichtet werden. Es ist keine Rückverstromung => Zwischenfazit: Biogas, Stromspeicher und eine flexible Elektrolyse bringen ausreichend Flexibilität ins Energienetz
  \item 100 \% Reduktion möglich. Es gibt eine expotentielle Kostenkurve, dennoch vertretbare Gesamtkosten
  \item Vergleich der Kosten zwischen 0 und 100 \% Reduktion
  \item Abwägung welcher Reduktionsfall im Vergleich zu den Kosten am sinnvollsten wäre?
  \item Umsetzbarkeit prüfen? Stehen wir wirklich in 30 Jahren bei Wasserstoff? Was macht der heutige Ausbau von EE? 
\end{itemize}

\todo{Im Fazit sollten wir noch einbauen, dass die Auslegungen/Maßnahmen Infrastruktur in den einzelnen Restriktionen zum Teil nicht für alle geeignet sind und es sinn macht heute sich an einem höheren RestriktionsZiel bezüglich dem zukünftigen Energiemodell zu orientieren.}

\todo{Etwas von oben übernehmen}

\todo{Welche Reduktion kann unser Energiesystem möglichst gut erreichen? Abwägung Reduktion/Kosten}



Die gewonnenen Erkenntnisse können mit anderen Simulationen, die andere Komponenten verwenden, hinsichtlich ihrer Gesamtkosten und Machbarkeit verglichen werden.