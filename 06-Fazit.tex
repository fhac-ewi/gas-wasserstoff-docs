\section{Fazit}
Anhand des Modells konnte dargestellt werden, dass die unterschiedlichen Szenarien zum Teil und insbesondere in Abhängigkeit der Treibhausreduzierungsziele, stark voneinander abweichen. Der Anwender kann sich mit Hilfe dieses Programms ein Bild über die erforderlichen Maßnahmen bis zur 100 \%-igen CO2-Neutralität machen. 

Festzuhalten ist der überproportionale Bedarf an Energie, der die Grundlage für das zu entwickelnde Energiesystem darstellt. Daneben stellen die PtX-Anlagen ein weiteres wichtiges Element zur Sektorkopplung dar. 
\newline
Die ambitionierten CO2 Ziele der Bundesregierung lassen aus heutiger Sicht eine nicht Erreichung mutmaßen. Umso wichtiger erscheinen die daraus gewonnenen Erkenntnisse, um den Wert annähernd zu erreichen. Der Stromerzeugungsmix wird als zwingende Voraussetzung ermittelt, so dass der Ausbau der erneuerbaren Energien einen sehr hohen Stellenwert hat. Somit wird die Zubaurate an EE in den Folgejahren signifikant steigen müssen. 

\todo{Mehr auf Speicher und Transport eingehen.}
\todo{Klimaziele}