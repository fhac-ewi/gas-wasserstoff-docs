\section{Fazit}
In einem CO\textsubscript{2}-neutralen Energiesystem muss die konventionelle Energiegewinnung durch Verbrennung von Erdgas vollständig zurückgebaut werden. Die durchgeführte Modellierung und Analyse verschiedener Reduktionsziele (vgl. Kapitel \ref{sec:reduktionsfaelle}) zeigt dabei detailliert die Auswirkungen des Rückbaus auf das Gesamtsystem auf.
Der Energieträger Erdgas ist aufgrund der vergleichsweise günstigen Kosten, der einfachen Speicherung und der hohen Flexibilität zur Anpassung der Angebotskurve an die momentane Nachfrage für das Gesamtsystem aus ökonomischer Sicht lukrativ. 

Mit steigendem Reduktionsziel können die erneuerbaren Erzeugungsanlagen aufgrund fehlender Flexibilität die Deckung der Nachfrage nicht mehr gewährleisten und es muss Biogas eingesetzt werden. Dieses verfügt über ähnliche Eigenschaften wie Erdgas, ist jedoch doppelt so teuer. Um zusätzliche Flexibilität bereitzustellen, werden Batteriespeicher und Pumpspeicherkraftwerke eingesetzt. Aufgrund der hohen Kosten können diese nicht die notwendige Kapazität für Reduktionsfälle $>=$ 90 \% bereitstellen.
Hierfür müssen zusätzlich Elektrolyseure als Flexsumer eingesetzt werden. Durch Salzkavernenspeicher kann die Wasserstofferzeugung von der Nachfrage entkoppelt werden und so für das Stromnetz Flex\-ibilität bereitstellen.  

In dem vorliegenden Energiemodell wird kein Power-to-Gas-to-Power eingesetzt. Die gro-ßen Salzkavernen könnten viel Energie speichern, jedoch ist der Wirkungsgrad mit ca. 44 \% gering und die Kosten sind deshalb hoch. Die Betriebskosten von Brennstoffzellen sind ebenfalls so hoch, dass diese nicht eingesetzt werden. 

Anstelle des weiteren Ausbaus von Speichersystemen ist es günstiger, erneuerbare Erzeugungsanlagen über Bedarf auszubauen. Im 100 \% Reduktionsfall wird dies besonders durch den massiven Ausbau von Photovoltaikanlagen deutlich. Während diese Anlagen im Reduktionsfall 90 \% zu 99,5 \% ausgenutzt werden, werden diese in einem CO\textsubscript{2}-neutralen Energiesystem nur zu 90,1 \% ausgenutzt.  

Bei kurzfristigen und mittelfristigen Infrastrukturmaßnahmen sollte immer die Langzeitplanung berücksichtigt werden, da mit steigendem Reduktionsziel gegenläufige Maßnahmen zur Erreichung eines möglichst kostengünstigen Energiesystems erforderlich sein kön-nen. In unserem Beispielszenario wird bis zu dem 95 \% Reduktionsfall die Wasserstoff-Pipeline\-kapazität sukzessive ausgebaut. Im 100 \%-Fall fällt diese um 30 \% ab, da eine Verteilung der Elektrolyseure in alle Cluster zur Stabilität des Stromnetzes erforderlich ist und damit die Transportkapazitäten von Wasserstoff zwischen den Clustern nicht mehr benötigt werden.

Ohne CO\textsubscript{2} Reduktionsziel würde das modellierte Energiesystem nach Optimierung der Gesamtkosten 40,6 Mrd. € pro Jahr kosten und 144,2 Mio. Tonnen CO\textsubscript{2} ausstoßen. Das CO\textsubscript{2}-neutrale Energiesystem würde nur 30 \% mehr (52,8 Mrd. €) pro Jahr kosten. 
Aus ökonomischer Sicht sind die Mehrkosten für ein CO\textsubscript{2}-neutrales Energiesystem zur Erreichung der vereinbarten Klimaschutzziele vertretbar.

Neben der ökonomischen Sichtweise muss auch die Machbarkeit in Betracht gezogen werden.
Das in dieser Projektarbeit erzeugte vereinfachte Energiesystemmodell spiegelt nur bedingt die Realität im 2050er Zielszenario wieder, da beispielsweise Wärmebedarf und Mobilität nicht betrachtet werden. Gerade die Sektorenkopplung mit der Möglichkeit von P2X einen hohen Einfluss auf die Anforderungen an ein zukünftiges Energiesystem. Aus diesem Grund kann das vereinfachte Energiesystemmodell keine belastbaren Ergebnisse liefern.

Die verabschiedeten Maßnahmen stellen einen ausreichenden Handlungsrahmen dar, um das Klima zu schützen. Damit die Ziele erreicht werden können, muss der Ausbau von erneuerbaren Energieanlagen in Deutschland deutlich beschleunigt werden. Dies könnte beispielsweise durch eine Reduzierung des bürokratischen Aufwands (Genehmigungen, Ausschreibungen) verkürzt werden.