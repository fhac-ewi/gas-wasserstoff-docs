\section{Analyse der Ergebnisse}
\todo{Unterüberschriften anpassen}
Im Folgenden erfolgt eine Gesamtbewertung für die einzelnen CO2-Restriktionen unter Berücksichtigung der jeweiligen Komponenten des Energiesystems. Im an-schließenden Kapitel 5 wird dann noch mal die herausstechende Besonderheit je Re-duktionsfall betrachtet. 

\subsection{Gewinnung von Energie: Ausbau Wind und Solar}
\begin{itemize}
  \item Wind + PV nehmen stark zu
  \item Gasimporte nehmen stetig ab. Bei 100\% Reduktion gar keine Erdgas Käufe mehr
\end{itemize}
Die unterschiedlichen Reduktionsziele stellen den Bedarf der Energieerzeugung aus erneuerbaren Energien dar. Insbesondere die Gewinnung aus Wind onshore, Wind offshore und Photovoltaik steigen signifikant, wobei der Anteil an Wind onshore klei-ner ausfällt als die beiden anderen genannten. Betrachtet man die installierte Leistung für Wind onshore für den 75 \% Rechenlauf liegt der Anteil bei 36 GW und reduziert sich bei 80\% auf 22 GW. Ab einer CO2 Reduktion von 80 \% nimmt der Anteil wieder zu und steigt im 100 \% Szenario auf 44 GW. Dahingegen wird bei Wind offshore ein signifikant steigender Anteil ermittelt. Im 75 \% Reduktionfall beträgt dieser 48 GW und steigt auf 81 GW bei einem Reduktionsfall von 90 \%. Dieser Wert bleibt bis zum 100 \% Szenario konstant. Ein besonders drastischer Anstieg ist bei der Energiege-winnung aus PV festzustellen. Beträgt dieser im Szenario 75 \% noch 29 GW steigt er bei allen weiteren Reduktionsfällen ab 85 \% überproportional bis hin zu 188 GW in-stallierter Leistung. 
GaskW gehen zurück. 



\subsection{Umwandlung von Energie: Elektrolyse}
\begin{itemize}
  \item Elektrolyse in allen Fällen gleich viel H2 im Jahr
  \item Mit steigenden Reduktionsfällen wird mehr Leistung in der Elektrolyse benötigt -> Spitzen von EE besser umsetzen
  \item Entsprechend den Speicherkapazitäten/Speichernutzung wird auch entsprechende Umwandlung benötigt
\end{itemize}


\subsection{Speicherung von Energie: Batterien und Salzkavernen}
\begin{itemize}
  \item Pumpspeicher in allen Reduktionsfällen am Kapazitätsmaximum, jedoch mit steigender Reduktion deutlich mehr Nutzung. 
  \item Karvernen werden bei geringen Reduktionsfällen nicht benötigt, da Gas importiert werden kann. Mit steigendem Reduktionsfall werden große Speicher (für Biogas und H2) benötigt. Diese Speicher können jedoch nur langsam Energie ein und ausspeichern -> Längerfristigere Einspeicherung
  \item Für Versorgungssicherheit werden Batteriespeicher benötigt. In den ersten Reduktionsfällen fast garnicht, da hier die Versorgungssicherheit über Gas geregelt ist. Bei 100 \% aufgrund des fehlenden Gases und der schwankenden Erzeugung werden große Kapazitäten benötigt.
  \item Generell alle Speicher steigen exorbitant an. (Außer Pumpspeicher - siehe Punkt 1)
\end{itemize}



\subsection{Übertragung von Energie: H2-Pipeline}
\begin{itemize}
  \item Kapazität für H2 nahezu konstant
  \item Bei 100\% Reduktion drastischer abfall -> Vermutlich wegen steigender Speichermöglichkeiten  
\end{itemize}

\subsection{Gesamtkosten (TAC)}





