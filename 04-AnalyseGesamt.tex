\section{Analyse der Ergebnisse}
\todo{Unterüberschriften anpassen}
Im Folgenden erfolgt eine Gesamtbewertung für die einzelnen CO2-Restriktionen unter Berücksichtigung der jeweiligen Komponenten des Energiesystems. Im an-schließenden Kapitel 5 wird dann noch mal die herausstechende Besonderheit je Re-duktionsfall betrachtet. 

\subsection{Gewinnung von Energie: Ausbau Wind und Solar}
\begin{itemize}
  \item Wind + PV nehmen stark zu
  \item Gasimporte nehmen stetig ab. Bei 100\% Reduktion gar keine Erdgas Käufe mehr
\end{itemize}
Die unterschiedlichen Reduktionsziele stellen den Bedarf der Energieerzeugung aus erneuerbaren Energien dar. Insbesondere die Gewinnung aus Wind onshore, Wind offshore und Photovoltaik steigen signifikant, wobei der Anteil an Wind onshore kleiner ausfällt als die beiden anderen genannten. Betrachtet man die installierte Leistung für Wind onshore für den 75 \% Rechenlauf liegt der Anteil bei 36 GW und reduziert sich bei 80\% auf 22 GW. Ab einer CO2 Reduktion von 80 \% nimmt der Anteil wieder zu und steigt im 100 \% Szenario auf 44 GW. Dahingegen wird bei Wind offshore ein signifikant steigender Anteil ermittelt. Im 75 \% Reduktionfall beträgt dieser 48 GW und steigt auf 81 GW bei einem Reduktionsfall von 90 \%. Dieser Wert bleibt bis zum 100 \% Szenario konstant. Ein besonders drastischer Anstieg ist bei der Energiege-winnung aus PV festzustellen. Beträgt dieser im Szenario 75 \% noch 29 GW steigt er bei allen weiteren Reduktionsfällen ab 85 \% überproportional bis hin zu 188 GW in-stallierter Leistung. 
GaskW gehen zurück. 



\subsection{Umwandlung von Energie: Elektrolyse}
\begin{itemize}
  \item Elektrolyse in allen Fällen gleich viel H2 im Jahr
  \item Mit steigenden Reduktionsfällen wird mehr Leistung in der Elektrolyse benötigt -> Spitzen von EE besser umsetzen
  \item Entsprechend den Speicherkapazitäten/Speichernutzung wird auch entsprechende Umwandlung benötigt
\end{itemize}


\subsection{Speicherung von Energie: Batterien und Salzkavernen}
% \begin{itemize}
%  \item Pumpspeicher in allen Reduktionsfällen am Kapazitätsmaximum, jedoch mit steigender Reduktion deutlich mehr Nutzung. 
%  \item Karvernen werden bei geringen Reduktionsfällen nicht benötigt, da Gas importiert werden kann. Mit steigendem Reduktionsfall werden große Speicher (für Biogas und H2) benötigt. Diese Speicher können jedoch nur langsam Energie ein und ausspeichern -> Längerfristigere Einspeicherung
%  \item Für Versorgungssicherheit werden Batteriespeicher benötigt. In den ersten Reduktionsfällen fast garnicht, da hier die Versorgungssicherheit über Gas geregelt ist. Bei 100 \% aufgrund des fehlenden Gases und der schwankenden Erzeugung werden große Kapazitäten benötigt.
%  \item Generell alle Speicher steigen exorbitant an. (Außer Pumpspeicher - siehe Punkt 1)
%\end{itemize}

Zur Erreichung der Reduktionsziele müssen vermehrt erneuerbare Energien, wie Wind und PV, eingesetzt werden. Anders als bei konventionellen Kraftwerken können diese Erzeuger nicht selbstbestimmt produzieren, sondern sind auf äußere Einflüsse angewiesen. Damit in dem Stromnetz jederzeit Angebot und Nachfrage sind Energiespeicher zwingend erforderlich. In dem Beispielszenario werden drei Speicherarten verwendet:
\begin{itemize}
  \item \textbf{Batteriespeicher} (kurzfristig)\\Aufgrund der schnellen Ein/Ausspeicherraten, die eine vollständige Entladung innerhalb einer Stunde ermöglichen, eignet sich der Batteriespeicher für den schnellen Leistungsausgleich. Durch die hohen Investitionskosten pro Kapazität und der vergleichsweise hohen Selbstentladung von 3 \% eignet sich der Speicher besonders für den kurzfristigen Einsatz.
  \item \textbf{Pumpspeicher} (mittelfristig)\\Eine vollständige Endladung des Speichers braucht ca. 6 Stunden. Der Speicher ist in dem Szenario bereits installiert.
  \item \textbf{Salzkarvernenspeicher} (langfristig)\\Karvernenspeicher können verhältnismäßig sehr viel Energie in Form von Gas speichern. Eine vollständige Entladung braucht ca. 20 Tage. Zudem gibt es (in unserem Beispiel) keine Selbstentladung, weshalb der Speicher für die langfristige Speicherung bestens geeignet ist. 
\end{itemize}

Wie auch die Wasserkraftwerke existieren die Pumpspeicher in dem Szenario bereits und werden für den täglich zum Ausgleich in den PEAK Stunden von 8-20 Uhr genutzt. \todo{Prüfen- wirklich Peak Stunden von 8 bis 20h? Oder vielleicht nur in den wirklichen Spitzen?} Mit steigender CO2 Reduktion sinkt der Anteil GuD-Kraftwerke als konventionelle Erzeuger und der Bedarf an einem sicheren Energielieferanten steigt. Dies wird durch die steigende Speichernutzung/Arbeit bestätigt, welche in Abbildung \todo{Abbildung erstellen und einfügen} zu sehen ist.
\todo{Was ist mit nachts? Werden dafür auch Wasserkraftwerke eingesetzt? Oder ist die Nachfrage so gering, dass Wind Erzeugung reicht?}

Ein steigender Speicherbedarf ist auch bei den Salzkarvernen zu erkennen, da bei steigenden CO2 Reduktionszielen weniger Erdgas importiert und verbrannt werden darf. Die Salzkarvernen werden zur Speicherung von Biogas und Wasserstoff genutzt und können bei saisonalen Engpässen in der Energieerzeugung (z.B. erzeugt PV im Winter deutlich weniger Energie als im Sommer) die Versorgung sicherstellen.

Um die starken, oftmals kurzfristigen, Schwankungen der erneuerbaren Energien auszugleichen werden Batteriespeicher eingesetzt. \todo{schwach, Abbildung vielleicht?}

\todo{Ggf. noch eine Abbildung entwicklung der Speicherkapazität zu Entwicklung der Speichernutzung?}

\subsection{Übertragung von Energie: H2-Pipeline}
\begin{itemize}
  \item Kapazität für H2 nahezu konstant
  \item Bei 100\% Reduktion drastischer abfall -> Vermutlich wegen steigender Speichermöglichkeiten  
\end{itemize}

\subsection{Gesamtkosten (TAC)}





