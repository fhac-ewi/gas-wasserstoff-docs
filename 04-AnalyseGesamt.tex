\section{Analyse der Ergebnisse}

\subsection{Gewinnung von Energie: Ausbau Wind und Solar}
\begin{itemize}
  \item Wind + PV nehmen stark zu
  \item Gasimporte nehmen stetig ab. Bei 100\% Reduktion gar keine Erdgas Käufe mehr
\end{itemize}


\subsection{Umwandlung von Energie: Elektrolyse}
\begin{itemize}
  \item Elektrolyse in allen Fällen gleich viel H2 im Jahr
  \item Mit steigenden Reduktionsfällen wird mehr Leistung in der Elektrolyse benötigt -> Spitzen von EE besser umsetzen
  \item Entsprechend den Speicherkapazitäten/Speichernutzung wird auch entsprechende Umwandlung benötigt
\end{itemize}


\subsection{Speicherung von Energie: Batterien und Salzkavernen}
\begin{itemize}
  \item Pumpspeicher in allen Reduktionsfällen am Kapazitätsmaximum, jedoch mit steigender Reduktion deutlich mehr Nutzung. 
  \item Karvernen werden bei geringen Reduktionsfällen nicht benötigt, da Gas importiert werden kann. Mit steigendem Reduktionsfall werden große Speicher (für Biogas und H2) benötigt. Diese Speicher können jedoch nur langsam Energie ein und ausspeichern -> Längerfristigere Einspeicherung
  \item Für Versorgungssicherheit werden Batteriespeicher benötigt. In den ersten Reduktionsfällen fast garnicht, da hier die Versorgungssicherheit über Gas geregelt ist. Bei 100 \% aufgrund des fehlenden Gases und der schwankenden Erzeugung werden große Kapazitäten benötigt.
  \item Generell alle Speicher steigen exorbitant an. (Außer Pumpspeicher - siehe Punkt 1)
\end{itemize}



\subsection{Übertragung von Energie: H2-Pipeline}
\begin{itemize}
  \item Kapazität für H2 nahezu konstant
  \item Bei 100\% Reduktion drastischer abfall -> Vermutlich wegen steigender Speichermöglichkeiten  
\end{itemize}

\subsection{Gesamtkosten (TAC)}





