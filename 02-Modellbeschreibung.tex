\section{Modellbeschreibung}
In diesem Kapitel wird das Energiesystemmodell für Deutschland bestehend aus Strom- und Gasinfrastrukturen eingeführt, welches im Rahmen dieser Projektarbeit hinsichtlich der Kosten unter Berücksichtigung verschiedener CO2-Reduktionsziele für das Jahr 2050 optimiert wird. In Tabelle \ref{tab:systemdesign2} sind die verwendeten Komponenten für das Energiesystemmodell ersichtlich.

\begin{table}[ht!]
    \begin{tabular}{|lcll|}
        \hline
        \multicolumn{1}{|l|}{Erneuerbare Energie Erzeugung} & \multicolumn{1}{l|}{Wind Onshore} & \multicolumn{1}{l|}{Wind Offshore} & Photovoltaik \\ \hline
        \multicolumn{1}{|l|}{Stromnetz AC}     & \multicolumn{3}{c|}{vorhanden}                \\ \hline
        \multicolumn{1}{|l|}{Stromnetz DC}     & \multicolumn{3}{c|}{vorhanden}                \\ \hline
        \multicolumn{1}{|l|}{H2 - Pipeline}    & \multicolumn{3}{c|}{vorhanden}                \\ \hline
        \multicolumn{1}{|l|}{Endenergiebedarf} & \multicolumn{3}{c|}{Strom und H2} \\ \hline
    \end{tabular}
    \caption{Komponenten des Energiesystems für Gruppe 3}
    \label{tab:systemdesign2}
\end{table}
\todo{Tabelle schöner gestalten. Ggf. auch nach Gewinnung, Speicherung, Verbrauch, Umwandlung, Transport aufteilen}
\todo{Weitere Komponenten wie Erdgas, Biogas, Wasserkraft, usw. auflisten}

Die Daten für das Energiesystemmodell in dem Betrachtungszeitraum \date{01.01.2050} bis \date{31.12.2050} stammen aus dem Beispiel ``Multi-regional Energy System Workflow'' des vom Institut IEK3 des Forschungszentrums Jülichs entwickelten Python-Framework FINE. \todo{Quelle}
Mithilfe des Frameworks FINE wird im weiteren Verlauf dieser Arbeit die Modellierung des Jahres 2050 in stündlicher Auflösung und Optimierung hinsichtlich der Gesamtkosten unter Einhaltung der gegebenen Rahmenbedingungen durchgeführt.

Das Modell bezieht sich ausschließlich auf die Region Deutschland. \todo{Bild Cluster Deutschland einbauen} Deutschland wird hierbei in acht Cluster unterteilt, welche sich durch unterschiedliche Kapazitätsgrenzen und ggf. Lastgänge für alle Komponenten zur Erzeugung, Umwandlung, Speicherung unterscheiden. Beispielsweise kann in Cluster vier, welches im Süden von Deutschland liegt, keine Offshore Windparks gebaut werden, da dieses Cluster keinen Zugang zum Meer hat. 

Bei der Modellierung muss in jedem Cluster der Energiebedarf von Strom und Wasserstoff zu jedem Zeitpunkt gedeckt sein. Für die Übertragung von Strom und Wasserstoff zwischen verschiedenen Clustern müssen Übertragungsnetze bzw. Pipelines mit entsprechenden Kapazitäten vorhanden sein. 

Nachfolgend wird auf die verwendeten Komponenten des Energiesystemmodells inklusive der technischen und ökonomischen Parameter eingegangen. 
 
\subsection{Energieträger}
In dem Energiesystem werden verschiedene Rohstoffe (Commodities) als Energieträger eingesetzt. 
Während die Nachfrage nur aus Strom und Wasserstoff besteht, können zusätzlich Methan und Biogas als Energieträger verwendet werden.

Zusätzlich wird in dem Modell CO2 als Commodity betrachtet und durch ein jährliches Limit beschränkt. Der Ausgangswert für die verschiedenen Reduktionsfälle wurde im Jahr 1990 ermittelt und liegt bei 366 Mio. Tonnen pro Jahr. 

\subsection{Energiegewinnung}
Zur Energiegewinnung können in dem Modell verschiedene Quellen genutzt werden.

Zum einen können bereits existierende Laufwasserkraftwerken mit Gesamtleistung von 3,8 GW zu geringen Betriebskosten von 790 Mio. € genutzt werden. Aufgrund der festen Betriebsrate (durch die Flüsse vorgegeben) können diese in dem Beispieljahr maximal 17.397 GWh Strom erzeugen.

Zum anderen können Methan und Biogas von außerhalb des modellierten Energiesystems zu festen Preisen von 33.100 € bzw. 54.090 € pro GW LWH importiert werden. Methan kann in unbegrenzter Menge importiert werden. \todo{Was hat es mit der OperationRateMax bei Biogas auf sich?} 
Damit diese Energieträger zur Deckung der Strom- bzw. Wasserstoffnachfrage verwendet werden können, sind entsprechende Umwandlungsanlagen notwendig. Diese werden in dem nachfolgenden Abschnitt beschrieben.

Eine weitere Möglichkeit zur Energiegewinnung von Strom sind erneuerbare Erzeugungsanlagen. In diesem Modell können Wind Onshore, Wind Offshore und Photovoltaikanlagen verwendet werden. Je Cluster wird die maximal installierbare Kapazität vorgegeben. 
Anders als konventionelle Kraftwerke können erneuerbare Erzeugungsnalgen nicht selbstbestimmt produzieren, sondern sind auf äußere Einflüsse (vorherrschende Winde und Sonneneinstrahlung) angewiesen. Um dies in dem Modell abzubilden, wird je Cluster und Anlagentyp eine maximale Betriebsrate in \% für jeden Zeitschritt festgelegt. Zusätzlich werden je Anlage ökonomische Parameter zur Bestimmung der Gesamtkosten in Abhängigkeit der installierten Kapazität angegeben. Diese können Tabelle \todo{ref} entnommen werden.

\todo{Tabelle mit invest, opex, interest rate, economic lifetime für die drei}

\subsection{Energieumwandlung}
Zur Energieumwandlung zwischen den Energieträgern werden Umwandlungsanlagen benötigt. So kann beispielsweise durch ein GuD-Kraftwerk Methan in Strom umgewandelt werden. In der Modellierung wird der Wirkungsgrad der Anlage sowie das dabei freigesetzte CO2 berücksichtigt. 

Neben den verschiedenen Gasanlagen, welche Gas in Strom umwandeln, stellt die Elektrolyse eine Möglichkeit zur Umwandlung von Strom in Wasserstoff zur Verfügung. Durch Brennstoffzellen ist zusätzlich eine Möglichkeit gegeben, um aus Wasserstoff Strom zu gewinnen.

Die technischen und ökonomischen Parametern aller Umwandlungsanlagen können Tabelle \todo{ref} entnommen werden.
\todo{Tabelle mit von-nach, Wirkungsgrad, freigesetztes CO2, invest, opex, interest, lifetime}

\subsection{Energiespeicherung}
Zu jedem Zeitschritt muss die Nachfrage durch ein Angebot gedeckt sein. Um dies zu gewährleisten werden je nach gewählten Erzeugungsanlagen Speicher zwingend benötigt.
In dem Beispielszenario können drei Speicherarten verwendet werden:
\begin{itemize}
    \item \textbf{Batteriespeicher} (kurzfristig)\\Aufgrund der schnellen Ein/Ausspeicherraten, die eine vollständige Entladung innerhalb einer Stunde ermöglichen, eignet sich der Batteriespeicher für den schnellen Leistungsausgleich. Durch die hohen Investitionskosten pro Kapazität und der vergleichsweise hohen Selbstentladung von 3 \% eignet sich der Speicher besonders für den kurzfristigen Einsatz.
    \item \textbf{Pumpspeicher} (mittelfristig)\\Eine vollständige Endladung des Speichers braucht ca. 6 Stunden. Der Speicher ist in dem Szenario bereits installiert.
    \item \textbf{Salzkarvernenspeicher} (langfristig)\\Karvernenspeicher können verhältnismäßig viel Energie in Form von Gas speichern. Eine vollständige Entladung braucht ca. 20 Tage. Zudem gibt es (in unserem Beispiel) keine Selbstentladung, weshalb der Speicher für die langfristige Speicherung bestens geeignet ist.   
\end{itemize}    

Analog zu der Energieerzeugung können die Speicher mit einer maximalen Kapazität je Cluster installiert werden. Die ökonomischen Parameter können der Tabelle \todo{ref} entnommen werden.

\todo{Tabelle mit ökonomischen Parametern}

\subsection{Energieübertragung}
Um die Energie zwischen den verschiedenen Clustern zu Übertragen, sind Übertragungsnetze erforderlich. 
\todo{Mehr schreiben}
\todo{Tabelle... I guess?}

- AC (bereits installiert)
- DC (bereits installiert)
- H2 Pipeline



\newpage
\todo{Ab hier alt}
Im praktischen Teil des Unterrichts im Fach Gas und Wasserstoffversorgungssysteme sollen zukünftige Energieverteilungsmodelle für Deutschland bei verschiedenen prozentualen CO2 Reduktionen im Vergleich zu 1990 mit damals emittierten 366 Mio. Tonnen simuliert werden. 

Basis der Simulationen stellt das vom Institut IEK3 des Forschungszentrums Jülichs bereitgestellten Framework „FINE“, welches die Programmiersprache Python implementiert und Energieszenarien mit Erzeugern, Verbrauchern und Speichern entwickeln und grafisch ausgeben kann.

Das verwendete Modell der Bundesrepublik wird dabei in acht Zonen als sogenannte Cluster von Null bis Sieben eingeteilt, welches in der Abbildung \ref{image:DEinCluster.png} zu sehen ist.
\todo{Cluster mit unterschiedlichen Nachfrage.. und Ausbaupotentialieln}

\image{DEinCluster.png}{Deutschland in 8 Cluster aufgeteilt}

\todo{Hier: Einleiten in Aufgabenstellung. Nicht so stark auf FINE einngehen}
FINE erzeugt nach Ablauf der Simulationen automatisch die Verteilung der Energiesysteme und Kopplung durch Pipelines und Stromtrassen inklusive einer visuellen Ausgabe der Ergebnisse.
Damit ein Vergleich der unterschiedlichen Teilnehmergruppen visualisiert und verglichen werden kann, benutzt jede Teilnehmergruppe in Ihrer Projektsimulation andere Energieversorgungs- und Verteilungskonzepte für Deutschland. Jedes Mitglied wird eine geforderte CO2 Reduktion zugewiesen, welche in den nachfolgenden Kapiteln von diesen präsentiert und erläutert werden.% In nachfolgender Tabelle XX sind die Reduktionen für die verschiedenen Simulationen zu sehen.


\begin{table}[ht!]
    \begin{tabular}{|lcll|}
        \hline
        \multicolumn{1}{|l|}{Erneuerbare Energie Erzeugung} & \multicolumn{1}{l|}{Wind Onshore} & \multicolumn{1}{l|}{Wind Offshore} & Photovoltaik \\ \hline
        \multicolumn{1}{|l|}{Stromnetz AC}     & \multicolumn{3}{c|}{vorhanden}                \\ \hline
        \multicolumn{1}{|l|}{Stromnetz DC}     & \multicolumn{3}{c|}{vorhanden}                \\ \hline
        \multicolumn{1}{|l|}{H2 - Pipeline}    & \multicolumn{3}{c|}{vorhanden}                \\ \hline
        \multicolumn{1}{|l|}{Endenergiebedarf} & \multicolumn{3}{c|}{Strom und Wasserstoff H2} \\ \hline
    \end{tabular}
    \caption{Energiesystemdesign}
    \label{tab:systemdesign}
\end{table}


Tabelle \ref{tab:systemdesign} zeigt das verwendete Konzept zur Simulationsausführung in FINE. Die Bereitstellung aus erneuerbaren Energien stellen hierbei Windkraft in On- und Offshore Konfiguration und Photovoltaik aus der direkten Sonnennutzung dar. Des weiteren existieren
Stromtrassen in Gleich- und Wechselstromkonfiguration und Pipelines zum Transport des verwendeten Wasserstoffs zur Verfügung. Die Energieverbräuche der einzelnen Sektoren werden in Form von Strom und direkter H2 Nutzung simuliert. Bestehende Wasserkraft, Methan und Biogasanlagen werden in die Simulation mit einbezogen und der Einkauf von
zusätzlichen Ressourcen in Form Gasförmigen Stoffen angenommen. Weil in der Realität der Ökonomische Gedanke stets eine Rolle spielt und keine Investitionen ohne die wirtschaftlichen Parameter durchgeführt werden zeigt nachfolgende Tabelle \todo{Verlinkung} die verwendeten und kalkulierten Kosten zur Installation und Inbetriebnahme der Energieressourcen in Deutschland. Das Framework „FINE“ bezieht diese Kenngrößen in die Simulation mit ein und berechnet die Gesamten volkswirtschaftlichen Investitionskosten, welche zukünftig zu treffen sind. 

%\image{tabelle_sonst.png}{Tabelle}

\begin{table}[ht!]
    \centering
    \begin{tabular}{|ccl|}
    \hline
    \multicolumn{1}{|c|}{\multirow{3}{*}{Wind Onshore}} & \multicolumn{1}{c|}{Investition}   & 1,1 Mrd. € / GW Leistung \\ \cline{2-3} 
    \multicolumn{1}{|c|}{}                              & \multicolumn{1}{c|}{Operation}     & 22 Mio. € / GW Leistung  \\ \cline{2-3} 
    \multicolumn{1}{|c|}{}                              & \multicolumn{1}{c|}{Lebenszyklus}  & 20 Jahre                 \\ \hline
    \multicolumn{1}{|c|}{\multirow{3}{*}{Wind Offshore}} & \multicolumn{1}{c|}{Investition}  & 2,3 Mrd. € / GW Leistung \\ \cline{2-3} 
    \multicolumn{1}{|c|}{}                              & \multicolumn{1}{c|}{Operation}     & 46 Mio. € / GW Leistung  \\ \cline{2-3} 
    \multicolumn{1}{|c|}{}                              & \multicolumn{1}{c|}{Lebenszyklus}  & 20 Jahre                 \\ \hline
    \multicolumn{1}{|c|}{\multirow{3}{*}{Photovoltaik}} & \multicolumn{1}{c|}{Investition}   & 650 Mio. € / GW Leistung \\ \cline{2-3} 
    \multicolumn{1}{|c|}{}                              & \multicolumn{1}{c|}{Operation}     & 13 Mio. € / GW Leistung  \\ \cline{2-3} 
    \multicolumn{1}{|c|}{}                              & \multicolumn{1}{c|}{Lebenszyklus}  & 25 Jahre                 \\ \hline
    \multicolumn{1}{|c|}{Gas / Methan}                  & \multicolumn{1}{c|}{Kaufpreis}     & 33.100 € pro GW Leistung \\ \hline
    \multicolumn{1}{|c|}{Biogas}                        & \multicolumn{1}{c|}{Kaufpreis}     & 54.090 € pro GW Leistung \\ \hline
    \end{tabular}
    \caption{Kostenparameter für Simulation}
    \label{tab:my-table}
\end{table}

\todo{Tabellen als Abbildung?}