\section{Analyse je Reduktionsfall}
Nachfolgend werden die einzelnen Reduktionsfälle im Detail analysiert.


\subsection{75 \%}

\subsection{80 \%}

\subsection{85 \%}
% Gesamtkosten: 43 Mrd Euro
% wandel von Natural gas auf Biogas (synthtische Herstellugn)

Für den Reduktionsfall von CO2 um 85\% würde das Gesamtsystem 43 Mrd. € kosten. Das System fokussiert sich dabei auf den Wechsel vom Erdgas auf Biogas, von den Importen über die Gasanlagen bis hinzu den Speichern. Dabei werden erstmalig Biogasanlagen und die Speichermöglichkeit von Biogas in Salzkavernen genutzt.

Die Kosten des Gasamtsystems schlüsseln sich dabei auf in 38 Mrd. € für die Ereugung, 4,3 Mrd. € für die Umwandlung, etwas unter 0,1 Mrd. € für die Speicherung und 0,1 Mrd. € für die Übertragung.
\newline
Im Vergleich zum Szenario mit 80\% CO2 Reduktion haben sich die Kosten für die Erzeugung und Speicherung erhöht, während die Kosten der Umwandlung niedriger sind. Die Kosten für die Übertragung ist in etwa gleich geblieben. 
Insgesamt gibt es eine Erhöhung der Gesamtsystemkosten von ca. 1 Mrd. €. Dies kann u.a. durch die Reduktion der Gasanlagen mit Importen, Gasanlagen und Gasspeicher mit sinkenden Kosten und den erforderlichen Investitionen in Importe, Anlagen und Speicher für Biogas mit steigenden Kosten erklärt werden.

% Erz: 38,7MRd
% Umw:4,3
% Speicherung: etwas unter 0,1Mrd
% Übertragung: 0,1 Mrd 
% zu 80\%: erz + spei erhöh, übertr ca. gleich, Umwandlung niedriger
%     nur 1MRD erhöhung wegen niedrigeren Kosten zur Umwandlung (deutlich weniger durch gasanlagen als mehr durch biogasanlagen)
%     erz: während importe von biohgas anfangen, wird Import von natural gas weniger

Das erstellte Energiesystem hat eine Gesamtkapazität von 200 GW Leistung durch Erzeugung und Umwandlung und eine Speicherkapazität von 510 GWh. Die Energie wird dabei zu 64\% auf den erneuerbaren Energien und zu 36\% aus Erdgas gewonnen. Mit 44\% hat die Erzeugung durch Wind Offshore die größte Beteiligung am Energiemix, jedoch hat Natural Gas mit 37\% auch noch einen großen Einfluss.
\newline
Es gibt einen steigenden Ausbau der Kapazität von Winderzeugung und PV-Erzeugung. Erstmalig wird in diesem Szenario Biogas importiert und Biogasanlagen genutzt. Weiterhin wird der Import und die Nutzung von Gasanlagen reduziert.
\newline
Genau wie beim Reduktionsfall von 80\% ist die Kapazität für die Elektrolyse mit 11,35 GW gleich geblieben.

% 200GW Erz + Umw
% 510GWh Speicherkapazität
% 36\%erdgas 64\%EE 
% Anteil Erdgas auf 36\% gesunken
% steigender Ausbau an Kapazität von Wind + PV
% gleich Elektrolyse
% erstmalige nutzung von biogasanlagen
% erstmaliger import von Biogas
% weitere Senkung von gasanlagen 
% weitere Senkung von natural gas importen


Die Strommasse, die nicht verbraucht werde, können kurzfristig in Batteriespeichern eingespeist werden. Diese wurden in diesem Szenario auf 4,3 GWh ausgebaut und deren Kapazität hat sich vervierfacht.
\newline
Insgesamt ist die Speicherkapazität um 651\% angestiegen. Die Speichernutzung für Pumpspeicher haben sich ungefähr um den Faktor 1,5 erhöht. Strom wird also öfter ein- und ausgespeichert, um dadruch die Extrema in der Lastkurve abzufangen. Zudem wird in diesem Energiesystem erstmalig Sakzkavernen zur Speicherung von Biogas mit einer Speicherkapazität von 427,4 GWh verwendet.

% starker Ansteig der Speicherung 
% steigung bei Batteriespeicher
% 4x kapa Batteriespeicher wie bei 80%
% keine salzkaverne Speicherkapazität
% erstmalige nutzung von biogas salzkaverne
% sprung bei Nutzung von Pumpspeicher

Die installierte Kapazität der H2-Pipeline ist konstant geblieben, da die Leistung der Elektrolyse und der nicht vorhandene Salzkaverne zur Speicherung von H2 nicht verändert wurde.
% leicht gesunken (nicht erwähnenswert) kapa h2 pipeline, 

Das Energiesystem im Reduktionsfall 85\% hat 
%Fazit
% Erdgas ist ab diesem Szenario nicht mehr größter Energieträger sondern WindOff
% Austausch von Natural Gas durch Biogas, Speicherung von Biogas in Kavernen + generell höhere Speichernutzung + kapazität -> verringerung der Emissionen von CO2 um 85%

\subsection{90 \%}
Eine CO2 Reduzierung um 90 \% würde 45 Mrd. Euro kosten. Der Reduktionsfall zeichnet sich dadurch aus, dass dieser keine kurzfristige Energiespeicherung mittels Batteriespeicher benötigt, das Kapazitätsmaximum für den Offshore Ausbau von Windenergie erreicht wird und erstmals Salzkarvernen zur Speicherung von Wasserstoff eingesetzt werden.

Die Gesamtkosten teilen sich in 39,6 Mrd. Euro für Erzeugung, 4,4 Mrd. Euro für Umwandlung, 1,3 Mrd. Euro für Speicherung und 0,1 Mrd. Euro für Transport/Übertragung auf. Gegenüber dem Reduktionsfall 85 \% haben sich die Kosten in allen Bereichen erhöht. Die Erhöhung von 2,2 Mrd. Euro ist besonders auf die erforderlichen Investitionen in Speicherung (+ 1,2 Mrd. Euro, Zuwachs von 14.800 \%) und Erzeugung (+ 0,8 Mrd. Euro, Zuwachs von 2 \%) zurückzuführen.

Das Energiesystem hält 228 GW Leistung zur Erzeugung und Umwandlung von Energie und 3.983 GWh Speicherkapazität vor. Die Nachfrage wird zu 75 \% aus erneuerbaren Energiequellen und zu 25 \% durch Erdgasimporte gedeckt. Die genaue Aufteilung des Energiemixes kann der Abbildung \ref{image:Energiemix90.png} entnommen werden. Die Offshore Winderzeugung wurde mit 81 GW Leistung auf die maximal mögliche Kapazität ausgebaut und wird nahezu vollständig (98,3 \%) über das gesamte Jahr ausgenutzt. 
Die Erdgasimporte und damit auch die GuD-Kraftwerke sind als einziger Erzeuger rückläufig.  

% TODO Bild hierhin setzen
\smallimage{Energiemix90.png}{Reduktionsfall 90 \% - Mix Energiegewinnung}

Falls der erzeugte Strom die aktuelle Nachfrage übersteigt, kann der Überschuss durch die deutlich ausgebaute Elektrolyse (18 GW Leistung, 99.383 GWh im Jahr) zu Biogas bzw. Wasserstoff umgewandelt werden. Sowohl Wasserstoff als auch das Biogas können in den Salzkarvernen mit einer Kapazität von insgesamt 3906 GWh gespeichert und zu einem späteren Zeitpunkt wieder zur Stromgewinnung (oder der Deckung des Wasserstoffbedarfs) eingesetzt werden. Die Salzkarvernen als langfristiger Energiespeicher werden in dem Beispieljahr ca. zehn Mal vollständig ge- und entladen. Die Pumpspeicher mit einer Kapazität von 77 GWh fangen die täglichen Lastspitzen ab und werden dafür ca. 122 Mal vollständig ge- und entladen.

Während der Erdgas Import um 33 \% gegenüber dem Reduktionsfall 85 \% sinkt, nimmt die Relevanz von Biogas und Wasserstoff als Energiespeicher zu. Infolgedessen erhöht sich auch die Pipelinekapazität für Wasserstoff geringfügig. 

Im Reduktionsfall 90 \% ist Erdgas weiterhin ein wichtiger Bestandteil, auch wenn der Anteil sinkt. Überschüsse der erneuerbaren Energien werden mittels Elektrolyse in Gas gewandelt und in Salzkarvernen gespeichert. Bei Bedarf wird dieses in GuD Kraftwerken wieder in Strom gewandelt werden. Die dabei entstehenden Emissionen sind so gering, dass das CO2 Reduktionsziel von 90 \% eingehalten werden kann. 

\subsection{95 \%}

\subsection{100 \%}
%\renewcommand{\reduktion}{100}
%\newpage
\section{CO\textsubscript{2} Reduktionsfall \reduktion\space\%}

\begin{tabularx}{0.5\textwidth} { 
  | >{\raggedright\arraybackslash}X 
  | >{\centering\arraybackslash}X | }
 \hline
 TAC [1e9 Euro/a] (Gesamtsystemkosten) & 42  \\
 \hline
 Installierte Leistung Wind-Onshore/ Wind-Offshore/ PV in den Regionen (je nach Gruppenvorgabe) [GW] & 42  \\
 \hline
 Installierte Leistung Elektrolyse [GW] & 42  \\
 \hline
 Speicherkapazität Batteriespeicher [GWh] & 42  \\
 \hline
 H2 Salzkavernen Speicherkapazität [GWh] & 42  \\
 \hline
 Installierte H2 Pipelinekapazität gesamt [GW] & 42  \\
 \hline
\end{tabularx}

