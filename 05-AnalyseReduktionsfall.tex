\section{Analyse je Reduktionsfall}
Nachfolgend werden die einzelnen Reduktionsfälle im Detail analysiert.
\todo{Prüfen, dass alle Zahlen mit Komma getrennt sind.}

\subsection{75 \%}

\subsection{80 \%}
Die Gesamtsystemkosten betragen 42 Mrd. Euro. Sie steigen im Vergleich zum 75 \%-Reduktionsfall leicht an. Insbesondere die Kosten der Wind-Offshore-Anlagen (19 Mrd. Euro) steigerten sich um 6 Milliarden. Diese Steigerung wird aber teilweise durch die sinkenden Erdgas-Importe ausglichen. Hierbei verringern sich die Kosten von 15 auf 12 Mrd. Euro. Dies lässt sich durch die Verringerung der CO2-Emissionen erklären. Die Kosten der restlichen Erzeugungsarten sinken um etwa 2 Mrd. Euro.

Werden die Gesamtsystemkosten nach Erzeugung, Umwandlung, Speicherung und Übertragung aufgeteilt, so liegt der Schwerpunkt bei der Erzeugung (37 Mrd. Euro). Die Umwandlung in den Elektrolyse-Anlagen ist mit 4,5 Mrd. Euro der zweitteuerste Kostenpunkt. Darauf folgt die Übertragung mit 123,7 Mio. Euro. Mit einem Anteil von 0,3 \% an den Gesamtkosten ist die Speicherung weit abgeschlagen.

Insgesamt beträgt die installierte Leistung der Erzeugungsanlagen 209 GW. Sie erfährt also eine Steigerung von 1,5 GW. Vermutlich lässt sich das auf die vermehrte Nutzung der Energiespeicher zurückführen. Die installierte Leistung Wind-Onshore geht auf 22,66 GW herunter. Die installierte Leistung Wind-Offshore nimmt stark zu. Ihr Wert beträgt 68,97 GW.  Weiter sinkt die installierte Leistung PV leicht auf 28,41 GW. Zudem bleibt die Elektrolyse-Leistung konstant auf 11,35 GW. 

Im gesamten Energiesystem werden 1.220.697,17 GWh/a erzeugt. Dabei überwiegen die erneuerbaren Energien mit einem Anteil von 53 \%. Erzeugung aus Erdgas ist im Vergleich zum 75 \%-Reduktionsfall um 10 \% zurückgegangen. Sie ist allerdings mit 47 \% noch im hohen Maße an der Energieproduktion beteiligt. Bei den regenerativen Energien stechen die Wind-Onshore Erzeugungsanlagen hervor. Diese haben zu 24,8 \% an der Gesamterzeugung und sind um 42,5 \% gegenüber dem 75 \%-Reduktionsfall angestiegen. 

Bei den Batteriespeichern gibt es eine Entwicklung. Ihre Speicherkapazität wächst auf 1,05 GWh an. Dies lässt sich vermutlich auf die geringere Nutzung der Gaskraftwerke zurückführen. Sie lassen sich nur noch vermindert für die kurzfristige Regelung im Netz einsetzten. Diese Aufgabe wird dann stärker von den Batteriespeichern übernommen. Ebenso finden die Pumpspeicherkraftwerke für die kurzen Lastspitzen Anwendung. Allerdings kann in den länger andauernden Dunkelflauten immer noch das importierte Erdgas eingesetzt werden. Dies führt dazu, dass weder H2- noch Biogas-Salzkarvenenspeicher eingesetzt. 

Die installierte H2-Pipelinekapazität steht unverändert bei 28,99 GW, da sich sowohl Kapazität der H2-Karvenenspeicher als auch die Elektrolyse-Leistung gleichgeblieben sind.

Sowohl bei den Kosten als auch bei der installierten Leistung hat die Offshore-Windkraft den größten Anteil. Hingegen gehen die Erzeugung aus Erdgas und die Importe des Erdgases zurück. Wobei 47 \% der erzeugten Energie weiter mit Erdgas produziert werden. Die restliche Erzeugung wird von erneuerbaren Energien gedeckt. Für kurzfristige Lastspitzen oder Engpässe werden vermehrt Batteriespeicher eingesetzt. Längerfristige Speicherung findet nicht statt, da keine Salzkarvenspeicher genutzt werden.


\subsection{85 \%}

\subsection{90 \%}
Eine CO2 Reduzierung um 90 \% würde 45 Mrd. Euro kosten. Der Reduktionsfall zeichnet sich dadurch aus, dass dieser keine kurzfristige Energiespeicherung mittels Batteriespeicher benötigt, das Kapazitätsmaximum für den Offshore Ausbau von Windenergie erreicht wird und erstmals Salzkarvernen zur Speicherung von Wasserstoff eingesetzt werden.

Die Gesamtkosten teilen sich in 39,6 Mrd. Euro für Erzeugung, 4,4 Mrd. Euro für Umwandlung, 1,3 Mrd. Euro für Speicherung und 0,1 Mrd. Euro für Transport/Übertragung auf. Gegenüber dem Reduktionsfall 85 \% haben sich die Kosten in allen Bereichen erhöht. Die Erhöhung von 2,2 Mrd. Euro ist besonders auf die erforderlichen Investitionen in Speicherung (+ 1,2 Mrd. Euro, Zuwachs von 14.800 \%) und Erzeugung (+ 0,8 Mrd. Euro, Zuwachs von 2 \%) zurückzuführen.

Das Energiesystem hält 228 GW Leistung zur Erzeugung und Umwandlung von Energie und 3.983 GWh Speicherkapazität vor. Die Nachfrage wird zu 75 \% aus erneuerbaren Energiequellen und zu 25 \% durch Erdgasimporte gedeckt. Die genaue Aufteilung des Energiemixes kann der Abbildung \ref{image:Energiemix90.png} entnommen werden. Die Offshore Winderzeugung wurde mit 81 GW Leistung auf die maximal mögliche Kapazität ausgebaut und wird nahezu vollständig (98,3 \%) über das gesamte Jahr ausgenutzt. 
Die Erdgasimporte und damit auch die GuD-Kraftwerke sind als einziger Erzeuger rückläufig.  

% TODO Bild hierhin setzen
\smallimage{Energiemix90.png}{Reduktionsfall 90 \% - Mix Energiegewinnung}

Falls der erzeugte Strom die aktuelle Nachfrage übersteigt, kann der Überschuss durch die deutlich ausgebaute Elektrolyse (18 GW Leistung, 99.383 GWh im Jahr) zu Biogas bzw. Wasserstoff umgewandelt werden. Sowohl Wasserstoff als auch das Biogas können in den Salzkarvernen mit einer Kapazität von insgesamt 3906 GWh gespeichert und zu einem späteren Zeitpunkt wieder zur Stromgewinnung (oder der Deckung des Wasserstoffbedarfs) eingesetzt werden. Die Salzkarvernen als langfristiger Energiespeicher werden in dem Beispieljahr ca. zehn Mal vollständig ge- und entladen. Die Pumpspeicher mit einer Kapazität von 77 GWh fangen die täglichen Lastspitzen ab und werden dafür ca. 122 Mal vollständig ge- und entladen.

Während der Erdgas Import um 33 \% gegenüber dem Reduktionsfall 85 \% sinkt, nimmt die Relevanz von Biogas und Wasserstoff als Energiespeicher zu. Infolgedessen erhöht sich auch die Pipelinekapazität für Wasserstoff geringfügig. 

Im Reduktionsfall 90 \% ist Erdgas weiterhin ein wichtiger Bestandteil, auch wenn der Anteil sinkt. Überschüsse der erneuerbaren Energien werden mittels Elektrolyse in Gas gewandelt und in Salzkarvernen gespeichert. Bei Bedarf wird dieses in GuD Kraftwerken wieder in Strom gewandelt werden. Die dabei entstehenden Emissionen sind so gering, dass das CO2 Reduktionsziel von 90 \% eingehalten werden kann. 

\subsection{95 \%}

\subsection{100 \%}
%\renewcommand{\reduktion}{100}
%\newpage
\section{CO\textsubscript{2} Reduktionsfall \reduktion\space\%}

\begin{tabularx}{0.5\textwidth} { 
  | >{\raggedright\arraybackslash}X 
  | >{\centering\arraybackslash}X | }
 \hline
 TAC [1e9 Euro/a] (Gesamtsystemkosten) & 42  \\
 \hline
 Installierte Leistung Wind-Onshore/ Wind-Offshore/ PV in den Regionen (je nach Gruppenvorgabe) [GW] & 42  \\
 \hline
 Installierte Leistung Elektrolyse [GW] & 42  \\
 \hline
 Speicherkapazität Batteriespeicher [GWh] & 42  \\
 \hline
 H2 Salzkavernen Speicherkapazität [GWh] & 42  \\
 \hline
 Installierte H2 Pipelinekapazität gesamt [GW] & 42  \\
 \hline
\end{tabularx}

