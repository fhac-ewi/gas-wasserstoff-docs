\section{Durchführung der Rechenläufe}
Zur Modellierung des Energiesystemmodells wurde das Framework FINE von dem Forschungszentrum Jülich eingesetzt.
Dieses Framework ermöglicht die einfache Erstellung eines Energiemodells mit verschiedenen Regionen und Energiearten. Alle benötigten Komponenten, wie Energiequellen, Verbraucher, Speicher, Umwandler und Netze, können mit technischen und ökonomischen Merkmalen hinzugefügt werden. 

Das FZ Jülich stellt diverse Beispiele bereit, die die Anwendung des FINE Frameworks exemplarisch zeigen. Unter anderem das Jupyter Notebook ``Multi-regional\_Energy\_System\_Workflow.ipynb'', welches ein Energiesystem mit acht Regionen modelliert und optimiert. Das Notebook wurde entsprechend der Aufgabenstellung (siehe Kapitel TODO) angepasst. Alle Rechenläufe werden mit der gleichen Datenbasis durchgeführt. Lediglich der Parameter ``zulässißger CO2 Ausstoß`` wird für die unterschiedlichen Rechenläufe verändert. 

Mithilfe des Gurobi Solvers wurde das Energiesystem hinsichtlich der Kosten unter Einhaltung der zuvor gesetzen Grenze für den CO2 Ausstoß optimiert. Das Ergebnis der Optimierung ist ein optimiertes Energiesystem, bei dem in jedem Zeitschritt die Nachfrage gedeckt und das CO2 Reduktionsziel eingehalten ist. Die Ausgabe von FINE ist vielseitig. Für jede Komponente und Region werden unter anderem die installierte Kapazität, der ``Fahrplan'' und die daraus resultierende Kosten ausgegeben.

Die Ausgabe der Optimierung wird in eine Excel-Datei exportiert, damit die Ergebnisse einfacherer analysiert und mit den Ergebnissen der anderen Reduktionsfällen verglichen werden können. 

Die Ausführung des Jupyter Notebooks für die sechs verschiedenen Reduktionsfälle wurde mithilfe eines Shell Scripts automatisiert. Das Script kopiert das Notebook, setzt das jährliche Limit für den CO2 Ausstoß entsprechend des Reduktionsfalles und führt das Notebook anschließend aus. Das Ergebnis ist eine Excel Datei je Reduktionsfall, welche anschließend zu einer übersichtlichen Excel zusammengefügt wird.
