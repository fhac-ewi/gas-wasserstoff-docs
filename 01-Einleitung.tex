\section{Einleitung}
\todo{Paul+Mel: Gewisse Wörter abkürzen? Abkürzungsverzeichnis}
\todo{Reduktionsfall, Reduktionsziel, Erneuerbare Energien, Photovoltaik, }

Diese Ausarbeitung beschäftigt sich im Rahmen des Moduls Gas- und Wasserstoffversorgungsstrukturen mit der Entwicklung eines Modells zur Optimierung eines nationalen Energiesystems. Dabei steht die Versorgung von Wasserstoff an zentraler Stelle. Zur Erzeugung soll dabei erneuerbarer Strom die Grundlage darstellen. Thematisch wird damit ein Bereich analysiert, der nicht nur die aktuellen Klimagegebenheiten, die Energiewirtschaft und Infrastruktur berücksichtigt, sondern auch den aktuellen politischen Rahmen.   

Gemäß dem Pariser Abkommen verabschiedete Deutschland seinen Beitrag zur Treibhausreduktion mit dem Klimaschutzplan 2050 und leitete somit ihren Beitrag zum internationalen Klimaschutz ein \todo{Quelle: \url{https://www.bmu.de/fileadmin/Daten_BMU/Download_PDF/Klimaschutz/klimaschutzplan_2050_bf.pdf}, Abruf vom 28.12.2021}. 
Die Umsetzung der Handlungsfelder erfolgt dabei mit unterschiedlichen Maßnahmen, wovon die in 2021 verabschiedete nationalen Wasserstoffstrategie eine der tragenden Säulen für die Folgejahre darstellt. \todo{Quelle: \url{https://www.bmbf.de/bmbf/de/forschung/energiewende-und-nachhaltiges-wirtschaften/nationale-wasserstoffstrategie/nationale-wasserstoffstrategie_node.html}, Abruf 28.12.2021} 

Um die verabschiedeten Ziele zu erreichen, werden die Maßnahmen mit dem übergreifenden Ziel der CO2-neutralen Stromversorgung verfolgt. Weitere Berücksichtigung findet daneben die Steigerung der Wasserstoffnachfrage, insbesondere in den Industriebereichen. Damit einher geht die Technologieentwicklung unter Berücksichtigung der vorhandenen Intrastruktur in Deutschland und der erforderlichen weiteren Schritte zur Gesamtsystemoptimierung im Energiesystem. 

Im Folgenden werden verschiedene CO2-Reduktionsbeschränkungen analysiert, die u. a. den Aspekt der Kosteneffizienz mitberücksichtigen. Daneben werden die Ansprüche an die technischen Voraussetzungen unter Anwendung eines vorgegebenen Energiesystemdesigns dargestellt.

\newpage
\todo{ab hier alt}
Weltweit haben in den letzten Jahrzehnten Naturkatastrophen aufgrund des menschenverursachten Klimawandels deutlich zugenommen. „Zwischen 2000 und 2009 waren es fünfmal so viele wie in den 70er Jahren“ \todo{Paul: ZITAT FLUTWELLE NRW 2021}
Eine statistische Auswertung hierzu ist in Abbildung \todo{Paul: Abbildung verlinken} zu sehen. Als Folgen des Klimawandels können Hitze- und Dürreperioden, Überschwemmungen, großflächige Waldbrände, schwerwiegende Stürme und häufige Extremwetterlagen beobachtet werden.
Damit das zusammenleben von Mensch und Planet Erde auch für die nächste Epoche gesichert ist, müssen neue Konzepte weg von fossilen Brennstoffen zu einer komplett auf erneuerbaren Energien konzipierten Energieversorgung entwickelt und in die Volkswirtschaften installiert
werden. Wichtig für die Konzeptfindung ist für die Entscheidungsträger dabei einen ganzheitlichen Überblick über die zu realisierenden Szenarien zu erhalten. Dies betrifft nicht nur die örtlichen Gegebenheiten für den Aufbau der Energiesysteme, sondern vor allem die entstehenden Kosten, die notwendige Logistik und notwendigen Ressourcen zur Installation und dem laufenden Betrieb der Anlagen. 

Im Lehrmodul Gas- und Wasserstoffversorgungsstrukturen wird als Projektarbeit eine solche Konzeptfindung für die zukünftige Energieversorgung der Bundesrepublik Deutschland einschließlich 100 \% CO2 Reduktion 2050 im Vergleich zu 1990 mit Simulationswerkzeugen Rechnerunterstützt mit Computer durchgeführt. 

Die nachfolgenden Seiten beschreiben die Ergebnisse der Simulationen und sollen allen Lesern einen schnellen Überblick über zukünftige Energieversorgungsszenarien plausibel darstellen. 

\todo{Besserer Übergang?}
\todo{Sollen wir einleiten, dass in verschiedenenn Gruppenarbeiten verschiedene Energiesysteme (von Komponenten her unterschiedlich) untersucht werden?}
\todo{Und innerhalb der Gruppe verschiedene Reduktionsziele betrachtet werden?}
\todo{Und dann einen besseren Übergang: Nachfolgend wird das Modell von Gruppe 3 beschrieben.}

\todo{Ziel laut Aufgabenstellung: Gesamtsystemoptimierung des Energiesystems inklusive Strom- und Gasinfrastrukturen Infrastrukturen unter Beachtung von gesetzten CO2 Reduktionen für das Jahr 2050. Das sollte in der Einleitung deutlich werden.}


\newpage
\todo{Version Paul}
Der fortschreitende Klimawandel und die daraus resultierenden Folgen gewann in den letzten Jahren immer mehr öffentliche Aufmerksamkeit und ist zu einem zentralen Thema in der Politik geworden. Deutschland und viele weitere Länder haben erkannt, dass ein Umdenken zum Schutz der Umwelt zwingend erforderlich ist. Das Übereinkommen von Paris wurde von 195 Staaten verabschiedet und soll durch eine Begrenzung der menschengemachten globalen Erwärmung auf 2 °C den Klimawandel bremsen. 

Damit Deutschland seinen Beitrag zum internationalen Klimaschutz erfüllen kann, verabschiedete das Bundeskabinett im November 2016 den Klimaschutzplan 2050. Dieser sieht eine weitgehende Treibhausgasneutralität für Deutschland bis zum Jahr 2050 vor \todo{Quelle: \url{https://www.bmu.de/fileadmin/Daten_BMU/Download_PDF/Klimaschutz/klimaschutzplan_2050_bf.pdf}, Abruf vom 28.12.2021}. 
Die Umsetzung der Handlungsfelder erfolgt dabei mit unterschiedlichen Maßnahmen, wovon die in 2021 verabschiedete nationalen Wasserstoffstrategie eine der tragenden Säulen für die Folgejahre darstellt. \todo{Quelle: \url{https://www.bmbf.de/bmbf/de/forschung/energiewende-und-nachhaltiges-wirtschaften/nationale-wasserstoffstrategie/nationale-wasserstoffstrategie_node.html}, Abruf 28.12.2021} 

Zur Erreichung der gesetzlich vereinbarten Ziele müssen weitreichende Maßnahmen getroffen werden.
Durch eine Energiewende soll die Energieerzeugung von fossilen Energieträgern auf nachhaltige und regenerative Energieträger umgestellt werden. Dies ist erforderlich, um die Ressourcen des Planten zu schonen und den Ausstoß von klimaschädlichem CO2 zu reduzieren. Damit die Energiewende gelingt ist es erforderlich, dass nicht nur die Energieerzeugung umgestellt, sondern das gesamte Energiesystem unter Berücksichtigung der vorhanden Infrastrukturen an die zukünftigen Begebenheiten angepasst wird.

Wie das zukünftige nationale Energiesystem aussieht, ist zum jetzigen Zeitpunkt unklar. Es sind verschiedene Systeme mit verschiedenen Energieträgern, wie beispielsweise Wasserstoff, und den dazugehörigen Technologien vorstellbar. 
Bei allen Überlegungen müssen neben den Klimazielen auch technische und wirtschaftliche Aspekte berücksichtigt werden. 
Um die Machbarkeit verschiedener Systeme zu analysieren und eine geeignete Lösung für die Energieversorgung der Zukunft zu finden, müssen Simulationen möglichst ganzheitlicher Modelle durchgeführt werden.

Im Rahmen der Projektarbeit im Modul Gas- und Wasserstoffversorgungsstrukturen soll ein nationales Energiesystem zur Versorgung von Wasserstoff auf Basis von erneuerbarem Strom erstellt werden. Ziel ist die Kostenoptimierung des Energiesystem bestehend aus Strom- und Gasinfrastrukturen unter Beachtung der gesetzten CO2-Reduktionen für das Jahr 2050. In dem nachfolgenden Kapitel wird das Modell beschrieben, welches in dieser Projektarbeit analysiert wird. Darauf aufbauend werden mehrere CO2 Reduktionsfälle untersucht und miteinander verglichen.