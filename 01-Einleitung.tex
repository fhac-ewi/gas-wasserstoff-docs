\section{Einleitung}
\todo{Paul+Mel: Gewisse Wörter abkürzen? Abkürzungsverzeichnis}
\todo{Reduktionsfall, Reduktionsziel, Erneuerbare Energien, Photovoltaik, }
Der fortschreitende Klimawandel und die daraus resultierenden Folgen gewann in den letzten Jahren immer mehr öffentliche Aufmerksamkeit und ist zu einem zentralen Thema in der Politik geworden. Deutschland und viele weitere Länder haben erkannt, dass ein Umdenken zum Schutz der Umwelt zwingend erforderlich ist. Das Übereinkommen von Paris wurde von 195 Staaten verabschiedet und soll durch eine Begrenzung der menschengemachten globalen Erwärmung auf 2 °C den Klimawandel bremsen. \todo{Quelle?}

Damit Deutschland seinen Beitrag zum internationalen Klimaschutz erfüllen kann, verabschiedete das Bundeskabinett im November 2016 den Klimaschutzplan 2050. Dieser sieht eine weitgehende Treibhausgasneutralität für Deutschland bis zum Jahr 2050 vor \todo{Quelle: \url{https://www.bmu.de/fileadmin/Daten_BMU/Download_PDF/Klimaschutz/klimaschutzplan_2050_bf.pdf}, Abruf vom 28.12.2021}. 
Die Umsetzung der Handlungsfelder erfolgt dabei mit unterschiedlichen Maßnahmen, wovon die in 2021 verabschiedete nationalen Wasserstoffstrategie eine der tragenden Säulen für die Folgejahre darstellt. \todo{Quelle: \url{https://www.bmbf.de/bmbf/de/forschung/energiewende-und-nachhaltiges-wirtschaften/nationale-wasserstoffstrategie/nationale-wasserstoffstrategie_node.html}, Abruf 28.12.2021} 

Zur Erreichung der gesetzlich vereinbarten Ziele müssen weitreichende Maßnahmen getroffen werden.
Durch die Energiewende soll die Energieerzeugung von fossilen Energieträgern auf nachhaltige und regenerative Energieträger umgestellt werden. \todo{Quelle} Dies ist erforderlich, um die Ressourcen des Planten zu schonen und den Ausstoß von klimaschädlichem CO2 zu reduzieren. Damit die Energiewende gelingt ist es erforderlich, dass nicht nur die Energieerzeugung umgestellt, sondern das gesamte Energiesystem unter Berücksichtigung der vorhanden Infrastrukturen an die zukünftigen Begebenheiten angepasst wird.

Wie das zukünftige nationale Energiesystem aussieht, ist zum jetzigen Zeitpunkt unklar. Es sind verschiedene Systeme mit verschiedenen Energieträgern, wie beispielsweise Wasserstoff, und den dazugehörigen Technologien vorstellbar. 
Bei allen Überlegungen müssen neben den Klimazielen auch technische und wirtschaftliche Aspekte berücksichtigt werden. 
Um die Machbarkeit verschiedener Systeme zu analysieren und eine geeignete Lösung für die Energieversorgung der Zukunft zu finden, müssen Simulationen möglichst ganzheitlicher Modelle durchgeführt werden.

Im Rahmen der Projektarbeit im Modul Gas- und Wasserstoffversorgungsstrukturen soll ein nationales Energiesystem zur Versorgung von Wasserstoff auf Basis von erneuerbarem Strom erstellt werden. Ziel ist die Kostenoptimierung des Energiesystem bestehend aus Strom- und Gasinfrastrukturen unter Beachtung der gesetzten CO2-Reduktionen für das Jahr 2050. In dem nachfolgenden Kapitel wird das Modell beschrieben, welches in dieser Projektarbeit analysiert wird. Darauf aufbauend werden mehrere CO2 Reduktionsfälle untersucht und miteinander verglichen.