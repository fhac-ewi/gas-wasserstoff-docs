\section{Einleitung}

Weltweit haben in den letzten Jahrzehnten Naturkatastrophen aufgrund des menschenverursachten Klimawandels deutlich zugenommen. „Zwischen 2000 und 2009 waren es fünfmal so viele wie in den 70er Jahren“ ZITAT FLUTWELLE NRW 2021
Eine statistische Auswertung hierzu ist in Abbildung XX zu sehen. Als Folgen des Klimawandels können Hitze- und Dürreperioden, Überschwemmungen, großflächige Waldbrände, schwerwiegende Stürme und häufige Extremwetterlagen beobachtet werden.
Damit das zusammenleben von Mensch und Planet Erde auch für die nächste Epoche gesichert ist, müssen neue Konzepte weg von fossilen Brennstoffen zu einer komplett auf erneuerbaren Energien konzipierten Energieversorgung entwickelt und in die Volkswirtschaften installiert
werden. Wichtig für die Konzeptfindung ist für die Entscheidungsträger dabei einen ganzheitlichen Überblick über die zu realisierenden Szenarien zu erhalten. Dies betrifft nicht nur die örtlichen Gegebenheiten für den Aufbau der Energiesysteme, sondern vor allem die entstehenden Kosten, die notwendige Logistik und notwendigen Ressourcen zur Installation und dem laufenden Betrieb der Anlagen. Im Lehrmodul Gas- und Wasserstoffversorgungsstrukturen wird als Projektarbeit eine solche Konzeptfindung für die zukünftige Energieversorgung der Bundesrepublik Deutschland einschließlich 100\% CO2 Reduktion 2050 im Vergleich zu 1990 mit Simulationswerkzeugen Rechnerunterstützt mit Computer durchgeführt. Die nachfolgenden Seiten beschreiben die Ergebnisse der Simulationen und sollen allen Lesern einen schnellen Überblick über zukünftige Energieversorgungsszenarien plausibel darstellen. 
TEST
