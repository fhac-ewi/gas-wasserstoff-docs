%%%%%%%%%%%%%%%%%%%%%%%%%%%%%%%%%%%%%%%%%%%%%%%%%%%%%%%%%%%%
%
% Pakete
%
%%%%%%%%%%%%%%%%%%%%%%%%%%%%%%%%%%%%%%%%%%%%%%%%%%%%%%%%%%%%
\usepackage[ngerman]{babel} % Sprache deutsch
\usepackage[left=2.50cm, right=2.50cm, top=3.50cm, bottom=2.50cm]{geometry} % Layout Geometrie
\usepackage[export]{adjustbox} % loads also graphicx
\usepackage[section]{placeins} % Floating
\usepackage{titleps} % Layout
\usepackage{parskip} % Remove indent on paragraphs but adds a little space
\usepackage{tabularx} % tables
\usepackage{multirow}

% Notizen (zum Deaktivieren [disable])
\usepackage[textwidth=25mm,textsize=footnotesize]{todonotes}
\setlength\marginparwidth{1in}

% Verlinkungen
\usepackage[
  breaklinks=false,
  linktocpage=false,
  linkcolor=black,
  citecolor=magenta,
  colorlinks=true,
  urlcolor=black
  ]{hyperref}

% Schriftart
\usepackage{charter}

% Quellen
\usepackage[
  backend   = biber,
  style     = numeric,
  sorting   = none,
 % backref   = true
]{biblatex}
\addbibresource{quellen.bib}
\DefineBibliographyStrings{german}{
  backrefpage = {referenziert auf Seite},
  backrefpages = {referenziert auf den Seiten},
  urlseen = {abgerufen am}
}
\DeclareFieldFormat{urldate}{\bibstring{urlseen}\space#1}
\setcounter{biburllcpenalty}{7000}
\setcounter{biburlucpenalty}{8000}

%%%%%%%%%%%%%%%%%%%%%%%%%%%%%%%%%%%%%%%%%%%%%%%%%%%%%%%%%%%%
%
% Style
%
%%%%%%%%%%%%%%%%%%%%%%%%%%%%%%%%%%%%%%%%%%%%%%%%%%%%%%%%%%%%
\newpagestyle{ruled}
{\sethead{\modulname}{- \thepage\space -}{Projektdokumentation - Gruppe 3}\headrule}
\pagestyle{ruled}


%%%%%%%%%%%%%%%%%%%%%%%%%%%%%%%%%%%%%%%%%%%%%%%%%%%%%%%%%%%%
%
% Befehle
%
%%%%%%%%%%%%%%%%%%%%%%%%%%%%%%%%%%%%%%%%%%%%%%%%%%%%%%%%%%%%
\newcommand{\image}[2]{
  \begin{figure}[!htp]
    \centering
    \includegraphics[max height=19cm,max width=0.9\textwidth,keepaspectratio]{images/#1}
    \caption{#2}
    \label{image:#1}
  \end{figure}
  \FloatBarrier
} % Beispiel: \image{Funny.PNG}{Dies ist eine Beschreibung}

\newcommand{\smallimage}[2]{
  \begin{figure}[!htp]
    \centering
    \includegraphics[max height=19cm,max width=0.6\textwidth,keepaspectratio]{images/#1}
    \caption{#2}
    \label{image:#1}
  \end{figure}
  \FloatBarrier
} % Beispiel: \smallimage{Funny.PNG}{Dies ist eine Beschreibung}


\newcommand{\tinyimage}[2]{
  \begin{figure}[!htp]
    \centering
    \includegraphics[max height=19cm,max width=0.4\textwidth,keepaspectratio]{images/#1}
    \caption{#2}
    \label{image:#1}
  \end{figure}
  \FloatBarrier
} % Beispiel: \tinyimage{Funny.PNG}{Dies ist eine Beschreibung}